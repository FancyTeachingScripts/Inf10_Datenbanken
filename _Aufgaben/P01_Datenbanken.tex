

\newcommand{\dbprojekt}[9]
{
\Datum{#1 bis #9}

\section{Rahmenbedingungen}
\begin{itemize}
    \item Bearbeitung in Teams bestehend aus 4 Personen
    \item Melden der Gruppen bis spätestens vor der ersten Deadline bei eurer Lehrkraft.
    \item Entwicklung nach dem Wasserfall-Modell:  \url{bycs.link/inf10-wasserfall} oder \\\url{de.wikipedia.org/wiki/Wasserfallmodell}
    \item Als Datenbanksoftware wird die Online-Software SQLution verwendet (kostenlos verfügbar unter: \UrlAndCode{sqlution.de}, Accounts bekommt ihr von eurer Lehrkraft).
    \item Zusätzlich zur Arbeit im Informatik-Unterricht wird auch \emphColB{Arbeit zuhause} notwendig sein. \emphColB{Arbeitsteilung mit ähnlicher Menge für jede:n ist erwünscht!}
    \item Empfehlung: Doku-Dateien auf BYCS-Drive speichern und dort bearbeiten.
    \item Alle Dokumenten-Abgaben im PDF-Format; Projektabgabe als ZIP-Datei des ganzen Projekts (Download in SQLution).
\end{itemize}

\section{Inhaltliche Anforderungen}
\begin{itemize}
    \item Datenbank mit mindestens 4 inhaltlichen Tabellen (Beziehungstabellen zusätzlich)
    \item alle 3 Kardinalitäts-Typen
    \item Mindestens: Erfüllung der 1. und 2. Normalform
    \item Bonus-Punkte: Erfüllung der 3. Normalform
    \item Auslesen von Daten durch den Nutzer (zur Erfüllung der Anforderungen) ausschließlich über SQL-Abfragen.
\end{itemize}

\section{Grober Zeitplan}

\begin{enumerate}
    \item \textbf{Woche ab #1:} Anforderungsanalyse und Einarbeitung in das Wasserfallmodell
    \item \textbf{Woche ab #2:} Finalisierung Anforderungsanalyse und Einarbeitung in Datenmodellierung\\
       $\rightarrow$  \emphColB{Abgabe: Lastenheft bis #2, 23:59 Uhr}
    \item \textbf{Woche ab #3:} Datenbankentwurf
       $\rightarrow$  \emphColB{Abgabe: Pflichtenheft bis #3, 23:59 Uhr}
    \item \textbf{Woche ab #4:} Umsetzung der Datenbank
    \item \textbf{Woche ab #5:} Umsetzung der Datenbank\\
       $\rightarrow$  \emphColB{Abgabe: Datenbank bis #6, #7}
    \item \textbf{Woche ab #6:} Tests + Dokumentation
    \item \textbf{Woche ab #8:} Abschluss der Dokumentation und Reflexion\\
       $\rightarrow$  \emphColB{Abgabe: Dokumentation und Reflexion bis #9, 23:59 Uhr}
\end{enumerate}

\section{Anforderung einzelner Phasen}

\subsection{Anforderungsanalyse}
\begin{itemize}
    \item Einarbeitung in das Wasserfallmodell 
    \item Entwurf der Projektidee
    \item Allgemeine Beschreibung des Produkts und Einsatz (\emphColB{Lastenheft})
    \item 5 funktionale und 2 nicht-funktionale Anforderungen (\emphColB{Lastenheft})
    \item Empfehlung: Formulierung einzelner TODOs auf einem Project Board.
\end{itemize}

\subsection{Datenbankentwurf}
\begin{itemize}
    \item Einarbeitung in Datenmodellierung (Redundanz, Konsistenz, Anomalien)
    \item Entwurf des Datenbankschemas als Klassendiagramm in SQLution.
    $\rightarrow$ \emphColA{Klassendiagramm} (\emphColB{Pflichtenheft})
    \item Entwurf User Interface (= gewünschte Eingabefelder und Dropdown Menüs + Aufbau Ausgabetabellen der SQL-Abfragen)
    $\rightarrow$ Festhalten mittels Skizzen/Beschreibung (\emphColB{Pflichtenheft})
    \item Erläuterung, welche Maßnahmen getroffen wurde, um Konsistenz sicherzustellen / Anomalien zu vermeiden (\emphColB{Pflichtenheft}).
    \item Wo möglich: Zuordnung der Inhalte zu den Anforderungen aus dem Lastenheft (\emphColB{Pflichtenheft}).
\end{itemize}

\subsection{Umsetzung}
\begin{itemize}
    \item Umsetzung des Plans aus dem Pflichtenheft
    \item Haltet euch möglichst genau an euer Pflichtenheft. Änderungen sollen nur notfalls vorgenommen werden und müssen begründet werden.
    \item Bearbeitung möglichst arbeitsteilig.
    \item \emphColA{Achtung: } Nach einer Änderung des Datenbankmodells muss eine neue Datenbank generiert und alle Datensätze neu eingetragen werden! Also erst gutes Daten-Design überlegen, dann umsetzen.
\end{itemize}

\subsection{Tests}
\begin{itemize}
    \item Kontinuierliches Testen während der Umsetzung.
    \item Überprüfung des Endprodukts zuerst anhand des \emphColA{Pflichtenhefts} anschließend anhand des \emphColA{Lastenhefts}
    \item Gesamtbeurteilung, ob das Projekt erfolgreich umgesetzt wurde.
    \item Festhalten aller Ergebnisse in einem Test-Protokoll (eine Auflistung, welche Anforderung wann von wem geprüft wurde und ob die Überprüfung erfolgreich war und, wenn nicht, wieso).
\end{itemize}

\subsection{Dokumentation}
\begin{itemize}
    \item Inhalt: \emphColB{Lastenheft, Pflichtenheft, Testprotokoll, Bedienungsanleitung, Finales Datenbankschema} (mit allen vorgenommenen Änderungen), \emphColB{alle SQL-Abfragen} (SQL-Code)
    \item Optisch ansprechendes Dokument!
\end{itemize}

\subsection{Bericht/Reflexion über...}
\begin{itemize}
    \item ...die einzelnen Projektphasen und aufgetretene Schwierigkeiten.
    \item ...während der Umsetzung erfolgte Änderungen gegenüber Pflichtenheft.
    \item ...was ihr wieder so/anders machen würdet.
    \item ...die erfolgte Arbeitsteilung.
\end{itemize}

\section{Benotung}
\subsection{Note 1: Produkt und Arbeitsphase}
\begin{itemize}
    \item Funktionalität der Datenbank, Erfüllung der Spezifikation (Anforderungen)
    \item Umsetzung des Wasserfallmodells
    \item Individuelle Beobachtung während Arbeitsphase
\end{itemize}

\subsection{Note 2: Dokumentation und Reflexion}
\begin{itemize}
    \item Verständlichkeit, Sprache
    \item Übereinstimmung mit dem Produkt
    \item differenzierte Reflexion der eigenen Arbeit
\end{itemize}
}