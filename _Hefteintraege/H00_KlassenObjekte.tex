\Hefteintrag{1.5}{Wdh: Klassen und Objekte}{
    \emphColA{\LoesungLuecke{Objekte}{5cm}} repräsentieren \emphColA{Gegenstände} in einem Computerprogramm. \emphColB{\LoesungLuecke{Klassen}{5cm}} sind der \emphColB{Bauplan}, der festlegt, welche \emphColC{Eigenschaften} (\emphColC{\LoesungLuecke{Attribute}{5cm}}) und \emphColor{red}{Fähigkeiten} (\emphColor{red}{\LoesungLuecke{Methoden}{5cm}}) einer bestimmten Objektart gespeichert werden sollen. Man stellt sie dar mit:

{
    %\vspace{0.1cm}
    \begin{minipage}[t]{0.3\textwidth}
        \centering
        \Large \emphColB{Klassenkarte}

        \vspace{0.1cm}
        \LoesungKaroTikz{
            \small
            \begin{class}[text width=\textwidth]{Person}{0,0}
                \attribute{String hobby}
                \attribute{int alter}
                \attribute{boolean hatHaustier}
                \attribute{String peinlichesErlebnis }
                \operation{void atmen()}
            \end{class}
        }{15}
        \Loesung{spitze Ecken}
    \end{minipage}
    \hfill
    \begin{minipage}[t]{0.3\textwidth}
        \centering
        \Loesung{
            \vspace{0.1cm}
            
            {\flushright
            $\leftarrow$ Klassenname}

            \vspace{-5pt}
            {\flushleft
            Objektname~:~Klassenname $\rightarrow$}

            \vspace{1cm}
            Attribute

            \vspace{2cm}
            {\flushright
            $\leftarrow$ Methoden}
        }
    \end{minipage}
    \hfill
    \begin{minipage}[t]{0.3\textwidth}
        \centering
        \Large \emphColA{Objektkarte}

        \vspace{0.1cm}
        \LoesungKaroTikz{
            \small
            \begin{object}[text width=0.95\textwidth]{p1 : Person}{0,0}
                \attribute{hobby = "Klettern"}
                \attribute{alter = 23}
                \attribute{hatHaustier = false}
                \attribute{peinlichesErlebnis = "..."}
            \end{object}
        }{15}
        \Loesung{runde Ecken}    
    \end{minipage}
    }
    
    \setstretch{1.0}
    \hinweis{Generell kann man Objektkarten mit oder ohne Methoden zeichnen, solange man es insgesamt einheitlich macht. Wir zeichnen sie daher immer \underline{ohne Methoden}.}
}