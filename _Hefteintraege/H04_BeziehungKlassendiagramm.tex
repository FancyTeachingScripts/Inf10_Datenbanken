\Hefteintrag{1}{Tabellenbeziehungen im Klassendiagramm}{
    \ifbeamer\else\vspace{12pt}\fi
    \LoesungLeer{}{2cm}
    \begin{tikzpicture}
        \begin{class}{TabelleA}{0,0}
            \attribute{int id}
            \attribute{String spalte1}
            \attribute{...}
        \end{class}
        \begin{class}{TabelleB}{11,0}
            \attribute{int id}
            \attribute{String spalte1}
            \attribute{...}
        \end{class}
        
        \Loesung{\uniAssociation{TabelleA}{n}{{fremdschlüssel}}{1}{TabelleB}}
    \end{tikzpicture}

    \LoesungLeer{{
    
        \begin{itemize}
        \color{\LFarbe}
            \item Beziehungspfeil immer vom Fremd- zum Primärschlüssel.
            \item 'fremdschluessel' ist eine Spalte der TabelleA, wird dort aber nicht eingetragen.
            \item Die Form der Pfeilspitze ist wichtig und muss genau so sein, da andere Spitzen andere Bedeutungen haben!
            \item Kardinalität an der Pfeilspitze ist immer 1 (bei Datenbanken), da in einer Spalte (eines Datensatzes) immer nur ein Wert stehen kann. 
        \end{itemize}
    }}{2cm}
}