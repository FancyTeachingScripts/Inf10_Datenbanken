\Hefteintrag{2}{Kardinalitäten}{
Die Kardinalität beschreibt, wie viele Objekte auf jeder Seite einer Beziehung stehen können. Es gibt folgende Arten:

\begin{itemize}
    \item \emphColA{1:1}, z.B. \LoesungLuecke{ein}{2cm} Häuptling pro Dorf, der auch nur in einem Dorf Häuptling ist.
    \item \emphColA{1:n}, z.B. jeder Bewohner wohnt in einem Dorf, das aber \LoesungLuecke{mehrere}{4cm} Bewohner hat.
    \item \emphColA{m:n}, z.B. \LoesungLuecke{beliebig viele}{5cm} Lehrer pro Schulklasse + \LoesungLuecke{beliebig viele}{5cm} Schulklassen pro Lehrer (in Datenbanken nicht direkt umsetzbar, dazu später mehr).
\end{itemize}
}