\Testart{Referate}
\Titel{Datenbankmodellierung}

\section*{Rahmenbedingungen}
\begin{itemize}
    \item Dauer: 15 $\pm$ 2 Minuten 
    \item Multimedia Präsentation (z.B. Powerpoint)
    \item alleine oder 2er Teams
    \item Einhaltung grundsätzlicher Aspekte von Referaten (siehe Bewertungsbogen)
    \item Präzise Quellenangaben für Informationen und Bilder (Google ist keine Quelle!) $\rightarrow$ möglichst genaue Links
    \item Bewertung anhand des Bewertungsbogens (siehe BYCS-Drive)
    \item Schulbuch als Orientierungshilfe aber nicht einzige Quelle!
    \item 1-2 DIN A4 Seiten Handout (als Hefteintrag für die Klasse, Lehrkraft druckt es aus)
    \item \textbf{Abgabe Präsentation und Handout spätestens 7:00 Uhr am Tag des Referats als Datei-Freigabe auf BYCS-Drive.}
\end{itemize}

\section*{Thema 1: Redundanz, Konsistenz und Anomalien}
\begin{itemize}
    \item Definitionen
    \item Arten von Anomalien
    \item Datenintegrität
    \item Gefahren (z.B. organisatorisch und Hacker)
    \item Was kann v.a. bei Tabellenbeziehungen schief gehen?
\end{itemize}

\section*{Thema 2: Datenmodellierung und Normalformen}
\begin{itemize}
    \item Wieso Normalformen? 
    \item Vor- und Nachteile
    \item Erste, zweite und dritte Normalform
    \item Beispiele, Anwendungsfälle
    \item Gefahren bei Nichtbeachtung
\end{itemize}